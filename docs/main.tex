\documentclass[12pt,a4paper]{report}

\usepackage[hungarian]{babel}
\usepackage{t1enc}
\usepackage[a4paper,top=2cm,bottom=2cm,left=3cm,right=3cm,marginparwidth=1.75cm]{geometry}
\usepackage{graphicx}
\usepackage{changepage}
\usepackage{array}
\usepackage[nolabel]{showlabels}
\usepackage{textpos} % textblock
\usepackage{float} % figure

% \usepackage{amsmath}
% \usepackage{amssymb}
\usepackage[colorlinks=true, allcolors=blue]{hyperref}
% \usepackage{mwe}

\graphicspath{ {./docs/res/} }

\begin{document}

\begin{titlepage}
    \centering
    \vspace{1cm}
    {\Large \textsc{A programozás alapjai 1\\(BMEVIEEAA00, 2024/25/1)\\  Nagyházi feladat}\par}
    \vspace{1.5cm}  
    {\huge\bfseries Shanon-Fano kódoló és dekódoló program\par}
    \vspace{2cm}
    {\Large\itshape készítette: \par Ferencz Péter\par (RFG7SN)}
    \vfill
    % \includegraphics[totalheight=3cm]{VIK.png}\par
    \includegraphics[totalheight=3cm]{BMEKicsi.png}\par
    Budapesti Műszaki és Gazdaságtudományi Egyetem\par
    Villamosmérnöki és Informatikai Kar\par
    Mérnökinformatikus Bsc\par
    {\large 2024 Október\par}
\end{titlepage}

\tableofcontents
\newpage

\chapter{Specifikáció}

\section{A program célja}
A program célja tetszőleges adat tömörítése majd ezek kitömörítése információvesztés nélkül.
Ennek megvalósítására a Shanon-Fano tömörítő algoritmust
\footnote{C. E. Shannon, „A Mathematical Theory of Communication”, 1948}
\footnote{Robert M. Fano, „The Transmittion of Information”, 1949}
alkalmazza.

\section{Felhasználói interakció}
A felhasználó két üzemmódot választhat ki a program futtatásakor: kódolás vagy dekódolás. Ezeket az első parancssori argumentumban a 
'kodol' és 'dekodol' kulcsszavakkal tudja kiválasztani.
\subsection{kódolás (kodol)}
\label{sec:encode}
Kódoló üzemmódban a bemenetet (lásd \nameref{ssec:in}) a Shanon-Fano kódoló algoritmust alkalmazva írja a kimenetre (lásd \nameref{ssec:out}) a kódolt adatot.
\begin{verbatim}
    program kodol --bemenet <fájl> --kimenet <fájl>
\end{verbatim}
\subsection{dekódolás (dekodol)}
\label{sec:decode}
Dekódoló üzemmódban a bemenetet (lásd \nameref{ssec:in}) a Shanon-Fano dekódoló algoritmust alkalmazva írja a kimenetre (lásd \nameref{ssec:out}) a dekódolt adatot.
\begin{verbatim}
    program dekodol --bemenet <fájl> --kimenet <fájl>
\end{verbatim}

\newpage
\section{A program által elfogadott kapcsolók}
\label{sec:flags}
A program futása során tetszőleges futtatást befolyásoló kapcsolókat (flageket) beállíthatunk.
Ezek sorrendje tetszőlegesen választható.
\subsection{Bemenet}
\label{ssec:in}
Parancssori megnevezés: \texttt{-{}-bemenet <forrásfájl>} \\
{\it Opcionális paraméter.}\\
Ha nincs megadva, de a program egy figyelmeztető üzenet kíséretében folytatja a lefutást. \\
A fájl méretétől és tartalmától független a program lefutása. \\
Az azt követő paraméter megadja a forrásfájl elérési útvonalát. Ha nincs megadva, stdin-ról kér be új sorral lezárt szöveget.

\subsection{Kimenet}
\label{ssec:out}
Parancssori megnevezés: \texttt{-{}-kimenet <célfájl>} \\
{\it Opcionális paraméter.}\\
Ha nincs megadva, de a program egy figyelmeztető üzenet kíséretében folytatja a lefutást.\\
A fájl méretétől és tartalmától független a program lefutása. \\
Az azt követő paraméter megadja a célfájl elérési útvonalát. Ha nincs megadva, stdout-ra írja ki a program a program kimenetét.

\subsection{Kódtábla}
Parancssori megnevezés: \texttt{-{}-kodtabla} \\
{\it Opcionális paraméter.}\\
Azt szabályozza, hogy a kódtáblát kiírja-e a program a standard kimenetre.

\subsection{Statisztika}
Parancssori megnevezés: \texttt{-{}-statisztika} \\
{\it Opcionális paraméter.}\\
Azt határozza meg, hogy a program kiírjon-e további számitásokat a program hatékonyságára vonatkozólag.\\
Az alábbi számítások történnek kiírásra: \\
\begin{itemize}
    \item Tömörítés mértéke: bemenet mérete a tömörített adat méretéhez képest
    \item Kódtábal mérete: Elgymástól eltérő kódok száma
    \item Kódok mérete: legrövidebb kód, leghosszabb kód, kódok átlagos mérete
    \item Fa mérete: A generált fa mérete
\end{itemize}

\subsection{Segítség}
Parancssori megnevezés: \texttt{-{}-help} \\
{\it Opcionális paraméter.}\\
A felhasználót tájékoztatja a program helyes használatáról. Ha ez a kapcsoló meg van adva, akkor a program nem ellenőrzi
a többi kötelező kapcsoló jelenlétét, kiírja a szöveget majd kódolás / dekódolás nélkül befejezi a futást.\\
Az alábbi szöveg íródik ki: \\
\begin{verbatim}    
    program [üzemmód] <...kapcsolók...>
    Üzemmód: kodol, dekodol
    Kapcsolók:
    --bemenet <forrásfájl>: Bemeneti fájl (ha üres akkor stdin)
    --kimenet <célfájl>: Bemeneti fájl (ha üres akkor stdout)
    --kodtabla <fájl>: A kódtábla fájl (kötelező)
    --statisztika: A tömörítés hatékonyságát értékelő statisztika (opcionális)
    --help: Ezt az üzenetet írja ki (opcionális)
\end{verbatim}


\section{A program kimenete}
Sikeres futtatás esetén a program a \nameref{sec:flags} pontban meghatározott viselkedés szerint működik.
Sikertelen futtatás esetén a konzolra kiíródik a probléma és egy nem nullás kilépési kóddal a program megáll. \\
A fájl ami generálódik a következőképpen épül fel:
Kódtábla karaktereinek száma: Hány darab karaktert és annak kódolását tartalmaz a kódtábla. Lehetséges értékei: 0-255 --> 1-256 \\
Illeszkedés hossza: A fájl végén hány darab 0 bit van a 8 bites fájlmentés kielégítéséhez. \\
Kódtábla, melynek minden eleme az alábbiakból épül fel:\\
\begin{itemize}
    \item Karakter: nyolc bit, melyet tömörítünk
    \item a karaktert reprezentáló kód hossza 8 biten
    \item a kód, mely nullás és eggyesek sorozata
\end{itemize}
Kódolt adat

\begin{figure}[H]
    \centerline{\includegraphics[width=\linewidth]{docs/res/encodedfilecontents.png}}
\end{figure}

% \section{A kódolás menete}
% C. E. Shannon, „A Mathematical Theory of Communication”, 1948
% Robert M. Fano, „The Transmittion of Information”, 1949


% \section{Kódolás folyamata}


% \section{Dekódolás folyamata}

% \chapter{minden más}

% \section{Megszorítások}
% A fájl értékei csak 0...255 értékeket vehetnek fel

% \section{A fejlesztést segítő programok}
% vscode \\
% hexdump \\
% xxd \\
% du \\

% C optimalizáció ki / be kapcsolása

% \section{buffer.c}
% The buffer holds 2 pointers, which enables it to be passed around as parameter
% this implementation also allows for arbitrary sized inputs to be read and outputs to be written

% \section{Felhasznált irodalom}
% https://web.archive.org/web/19980715013250/http://cm.bell-labs.com/cm/ms/what/shannonday/shannon1948.pdf \\
% https://hcs64.com/files/fano-tr65-ocr-only.pdf\\
% \href{https://man7.org/linux/man-pages/index.html}{Linux man-pages project}\\
% \href{https://infoc.eet.bme.hu/}{BME Infoc segédanyagok}\\


\end{document}