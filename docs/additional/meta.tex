\chapter{Meta}
\section{Források, felhasznált irodalom}
\begin{itemize}
    \item Wayback Machine: C. E. Shannon, „A Mathematical Theory of Communication”, 1948 (\href{https://web.archive.org/web/19980715013250/http://cm.bell-labs.com/cm/ms/what/shannonday/shannon1948.pdf}{https://web.archive.org/web/19980715013250/http://cm.bell-labs.com/cm/ms/what/shannonday/shannon1948.pdf})
    \item Halley's Comet software: Robert M. Fano, „The Transmittion of Information”, 1949 (\href{https://hcs64.com/files/fano-tr65-ocr-only.pdf}{https://hcs64.com/files/fano-tr65-ocr-only.pdf})
    \item Linux man pages online: (\href{https://man7.org/linux/man-pages/index.html}{https://man7.org/linux/man-pages/index.html})
    \item BME InfoC: (\href{https://infoc.eet.bme.hu}{https://infoc.eet.bme.hu})
\end{itemize}

\section{Felhasznált segédprogramok}
\begin{itemize}
    \item Fejlesztői környezet: Visual Studio Code (\href{https://code.visualstudio.com/}{https://code.visualstudio.com/})
    \item C compiler: GNU Compiler Collection (GCC) (\href{https://gcc.gnu.org/}{https://gcc.gnu.org/})
    \item Projekt fordítása: Make (\href{https://www.gnu.org/software/make/}{https://www.gnu.org/software/make/})
    \item Dokumentáció: Doxygen (\href{https://www.doxygen.nl/}{https://www.doxygen.nl/})
\end{itemize}